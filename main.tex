\iffalse
	前言:
		参考自一个祖传的模板,以及https://github.com/LeyuDame/BNUCV/tree/main 上的BNU的latex简历模板中的代码。
		注意要用XeLaTeX编译链进行编译,且要进行三次编译才能显示照片。问就是LaTeX的锅。
		vscode+latex的话,配置的json中"latex-workshop.latex.recipes"添加:
		{
            "name": "XeLaTeX*3",
            "tools": [
                "xelatex",
                "xelatex",
                "xelatex"
            ]
        }
		即可使用三次XeLaTeX编译。

		LaTeX+VScode怎么配置看https://www.zhihu.com/column/p/166523064。

        每个章节的格式都能混着用,顺序都可以变,只是给了个例子。
        比如你找工作,可以把技能那部分往前挪。
        比如你竞赛经历很多,你就往前挪。
        比如你觉得“其他”有点多余,就删了。

        主要贡献还是把原本word的那个模板页眉页脚和背景完美加进来了。
        LaTeX排版就是很整齐,强迫症狂喜。


        记得要三次XeLaTeX编译!!!
        记得要三次XeLaTeX编译!!!
        记得要三次XeLaTeX编译!!!
        这个很重要,所以说三遍!

        你可以试试两次的,好像某些环境两次编译也能显示图像,但最好还是三次编译。
\fi

\documentclass[10pt]{article}
\usepackage{xltxtra}
\usepackage{bookmark}
\usepackage{hyperref}
\hypersetup{hidelinks}
\usepackage{url}
\urlstyle{tt}
\usepackage{multicol}
\usepackage{xcolor}
\usepackage{calc}
\usepackage{graphicx}
\usepackage{tikz}
\usetikzlibrary{calc}
\usepackage{fontspec}
\usepackage{xeCJK}
\usepackage{relsize}
\usepackage{xspace}
\usepackage{fontawesome}
\usepackage{titlesec}
\usepackage{enumitem}
\usepackage{siunitx}
\usepackage{amssymb}
\usepackage{tabularx}
\usepackage{multicol}
\usepackage{fontspec}


% 一些小设置,参考自https://github.com/LeyuDame/BNUCV/tree/main
\CJKsetecglue{}							            % 取消中文字符与数字之间的间隔
\protected\def\Cpp{{C\nolinebreak[4]\hspace{-.05em}\raisebox{.28ex}{\relsize{-1}++}}\xspace}	% 这是个更好看的C++写法,你直接写C++的话,+号会很大,可以使用\Cpp来代替
\setlength{\parindent}{0pt}							% 取消全局段落缩进
\pagenumbering{gobble}								% 取消页码显示
%\setlist{noitemsep}									% 禁用列表中项目之间的额外垂直间距,但保留列表周围的间距
%\setlist{nosep}										% 禁用列表中项目之间的额外垂直间距及列表周围的间距
\setlist[itemize]{topsep=0em, leftmargin=*}		% 增加了itemize顶部间距
\setlist[enumerate]{topsep=0em, leftmargin=*}	% 增加了enumerate顶部间距

\titleformat{\section}					    % 将原标题前面的数字取消了
  {\LARGE\bfseries\raggedright} 		      % 字体改为LARGE,bold,左对齐
  {}{0em}                      			  % 可用于添加全局标题前缀
  {}                           			  % 可用于添加代码
  [{\color{CQMU_Green}\titlerule}]            % 标题下方加一条线
\titlespacing*{\section}{0cm}{*1.2}{*1.2}	% 标题左边留白,上方1.2倍,下方1.2倍

\titleformat{\subsection}				    % 将原二级标题前面的数字取消了
  {\large\bfseries\raggedright} 		      % 字体改为large,bold,左对齐
  {}{0em}                      			  % 可用于添加全局二级标题前缀
  {}                           			  % 可用于添加代码
  []
\titlespacing*{\subsection}{0cm}{*1.2}{*1.2}% 二级标题左边留白,上方1.2倍,下方1.2倍

% 页面大小与页边距,按需求调整
\usepackage[
	a4paper,
	left=1.2cm,
	right=1.2cm,
	top=1.5cm,
	bottom=1cm,
	nohead
]{geometry}

% 中文字符间距
\renewcommand{\CJKglue}{\hskip 0.05em}

% 英文字体
\setmainfont[
    Path=fonts/,
    Extension=.ttf,
    BoldFont=* Bold,
]{Microsoft Yahei}
% 中文字体
\setCJKmainfont[
    Path=fonts/,
    Extension=.ttf,
    BoldFont=* Bold,
]{Microsoft Yahei}

% 主题色
% 重医绿
\definecolor{CQMU_Green}{RGB}{  2, 90, 92}

% 这里把表格的行间距调成1.2倍了
\renewcommand{\arraystretch}{1.2}
% 这里把正文的行间距调成1.2倍了
\linespread{1.2}

%%%%%%%%%%%%%%%%%%%%%%%%%%%%%%%%%%%%%%%%%%%%%%%%%%%%%%%%%%%%%%%%%%%%%%%%%%%%%%%%%%%%%%%%%%%%%%%%%%%%%%%%%%%%%%%%%%%%%%%%%%%%%%%%%%%%%%%%%%%%%%%%%%
%    !!!!!!!! 记得改这里 !!!!!!!!
%%%%%%%%%%%%%%%%%%%%%%%%%%%%%%%%%%%%%%%%%%%%%%%%%%%%%%%%%%%%%%%%%%%%%%%%%%%%%%%%%%%%%%%%%%%%%%%%%%%%%%%%%%%%%%%%%%%%%%%%%%%%%%%%%%%%%%%%%%%%%%%%%%
% 学院
\newcommand{\school}{人工智能医学学院 | College of Medical Artificial Intelligence}
% 也可以不写英语
%\newcommand{\school}{电子信息学院}
% 联系方式
\newcommand{\contact}
{
    \small              % 换了更小的字号
    % \footnotesize       % 这比上面的小一号
    \scriptsize         % 这比上面的再小一号
    \textcolor{white}
    {
        \faEnvelope \quad \href{losmosga@foxmail.com}{abc@foxmail.com}    % 邮箱,前面的超链接可以直达邮箱软件
        \hspace{4em}    % 这里可以调间距
        \faWechat \quad WeChat             % 微信
        \hspace{4em}    % 这里可以调间距
        \faPhone \quad xxxxxxxxxxx             % 手机号
        \hspace{4em}    % 这里可以调间距
        \faGithub \quad \href{https://github.com/}{https://github.com/}         % github
    }
}



%%%%%%%%%%%%%%%%%%%%%%%%%%%%%%%%%%%%%%%%%%%%%%%%%%%%%%%%%%%%%%%%%%%%%%%%%%%%%%%%%%%%%%%%%%%%%%%%%%%%%%%%%%%%%%%%%%%%%%%%%%%%%%%%%%%%%%%%%%%%%%%%%%
%    !!!!!!!! 这里开始就是正文了 !!!!!!!!
%%%%%%%%%%%%%%%%%%%%%%%%%%%%%%%%%%%%%%%%%%%%%%%%%%%%%%%%%%%%%%%%%%%%%%%%%%%%%%%%%%%%%%%%%%%%%%%%%%%%%%%%%%%%%%%%%%%%%%%%%%%%%%%%%%%%%%%%%%%%%%%%%%
\begin{document}
	% 如果有多页简历,请把页眉页脚和背景复制粘贴到第二页的内容之前
	% 页眉,校徽,学院名
 % ------------------------------------------------------------------------
	\begin{tikzpicture}[remember picture, overlay]
		\node[anchor=north, inner sep=0pt](header) at (current page.north){
			\includegraphics[width=\paperwidth]{images/cqmu-header-green.png}
		};
		\node[anchor=west](school_logo) at (header.west){
			\hspace{0.5cm}
			\includegraphics[width=0.35\textwidth]{images/cqmu-logo-white.png}
		};
		\node[anchor=east](school_name) at(header.east){
			\textcolor{white}{\textbf{\school}}
			\hspace{0.5cm}
		};
	\end{tikzpicture}
	\vspace{-4em}

	% 页脚,联系方式
	\begin{tikzpicture}[remember picture, overlay]
		\node[anchor=south, inner sep=0pt](footer) at (current page.south){
			\includegraphics[width=\paperwidth]{images/cqmu-footer.png}
		};
        % 联系方式
        \node[anchor=center] at(footer.center){\contact};
	\end{tikzpicture}
	
	% 背景
	\begin{tikzpicture}[remember picture, overlay]
		\node[opacity=0.1] at(current page.center){
			\includegraphics[width=0.8\paperwidth, keepaspectratio]{images/cqmu-logo.eps}
		};
	\end{tikzpicture}
% ---------------------------------------------------------------------
	% 个人信息
    \begin{figure}[h]
        % 左半边,信息,比例占行宽86%,可以自己调
        \begin{minipage}{0.80\textwidth}
            \section{\makebox[\widthof{\faUser}][c]{\color{CQMU_Green}{\faUser}}\quad 个人信息}
            \begin{tabularx}{\linewidth}{p{\widthof{出生日期:}}Xp{\widthof{政治面貌:}}X}
                姓名: & 张三 & 性别: & 男|女 \\
                出生日期: & 2003年11月14日 & 政治面貌: & 群众|到群众中去 \\
                电话: & xxxxxxxxxx &地址: & 广东省深圳市\\ 电子邮箱: &  abc@foxmail.com \\
                    
                %% 想多加几行的话,就按上面的格式自行补充
                %% 想加粗的话\textbf{}
                %% 想多加几列的话,把\begin{tabularx}{\textwidth}{这里}的内容改一下,可以自己搜一下tabularx怎么用,也可以问gpt/文心一言/讯飞。
            \end{tabularx}
        \end{minipage}
        % 右半边,照片,比例占行宽16%,可以自己调
        % images/example_avatar.png 替换成你证件照的路径。
        \begin{minipage}{0.16\textwidth}
            \includegraphics[width=\linewidth]{images/example_avatar.png}
        \end{minipage}
        % 尽量留至少1%的间距,不然会换行
    \end{figure}
\vspace{-1em}
\section{\makebox[\widthof{\faHeart}][c]{\color{CQMU_Green}{\faHeart}}\quad 求职意向}
\vspace{-1em}
\begin{table}[h!]
        \begin{tabularx}{\textwidth}{XXp{\widthof{2021年 -- 预计2025年7月毕业}}}
            
             \textbf{职位:} 计算机工程师 &\textbf{期望行业:} 35岁无忧& \textbf{期望工作地点:} 地球\\
             
    
            % 这里哪个高就加粗哪个,哪个不想放就留白 =w= 比如你综测高GPA低,你就只放综测和综测排名就好。
        \end{tabularx}

        
    \end{table}
	% 教育背景
 %    	\faGraduationCap这类\fa开头的都是font awesome里的logo,想换成其他logo的话,可以看一下附带的fontawsome.pdf,自行替换。
		% \section{\makebox[\widthof{     这里!    }][c]{\color{CQMU_Green}{     和这里!    }}\quad 标题}
	\section{\makebox[\widthof{\faGraduationCap}][c]{\color{CQMU_Green}{\faGraduationCap}}\quad 教育背景}
	\vspace{-1em}
    \begin{table}[h!]
        \begin{tabularx}{\textwidth}{XXp{\widthof{2021年 -- 预计2025年7月毕业}}}
            重庆医科大学xx学院 & xx专业 & 2021年 -- 预计20xx年7月毕业\\
             \textbf{GPA:4.99/5.0} & \textbf{基础素质排名: T0/114} & \textbf{综合素质排名: T0/114} \\
             \textbf{核心课程(100):}& & \\
             \small 如何与产品经理共用大脑(92)& \small \textbf{全栈(全干)开发} & \small Vim的哲学 (92) \\
             \small{数据库删库跑路指南(94)} & \small CPP十年入门(99)& \small \textbf{PHP世一语(98)}\\ \small \textbf{网络拓扑回环教程(80)}&\small \textbf{Kali Linux我看邢(89)}&\small \textbf{如何点亮一颗电容(91)}\\
             
    
            % 这里哪个高就加粗哪个,哪个不想放就留白 =w= 比如你综测高GPA低,你就只放综测和综测排名就好。
        \end{tabularx}
        

        
    \end{table}
    
    % 项目经历(找导师一般都看中这个),可以改成“科研经历”
    %     \faGears 这是齿轮,适合机械类,我电信的也喜欢齿轮,就用这个了
    %     \faFlask 这是烧瓶,适合生化类
    %     \faLaptop 这是个笔记本电脑,适合计算机类
    %     \faUsers 这是三个人,适合商科
    \section{\makebox[\widthof{\faGears}][c]{\color{CQMU_Green}{\faLaptop}}\quad 实践经历}
    \vspace{0.5em}
    % 小技巧,老师想看的重点加粗,比如商科类的一般更想看到数字,工科类的更想看到技术
    \subsection{\Large 实习经历}
    \vspace{0.5em}
    \textbf{\uppercase\expandafter{\romannumeral1}. 我去的第一家实习机构( •̀\ ω •́\ ): 我是练习时长xx实习生
    \quad{\color{CQMU_Green}\faMapMarker\textit{哎哟你在哪}}}  \hfill 2022年6月-2022年9月
    \vspace{0.4em}
    \begin{itemize}[itemsep=1pt, leftmargin=2em]
        \item[\Large $\bullet$] \small 我学习到了如何更优雅地\textbf{唱、跳、rap},为公司带来了\textbf{2.5亿}的流量;
        \item[\Large $\bullet$] \small 掌握了Spring框架中\textbf{炒粉模块}的类加载器与\textbf{动作模块动态注入}的关系和模式;
    \end{itemize}

    \vspace{1em}
    \textbf{\uppercase\expandafter{\romannumeral2}. 狠赚笔有限公司: Java炒粉实习生
    \quad{\color{CQMU_Green}\faMapMarker\textit{广东深圳}}}  \hfill 2023年6月-2024年2月
    \vspace{0.4em}
    \begin{itemize}[itemsep=1pt, leftmargin=2em]
        \item[\Large $\bullet$] \small 项目简介:负责开发一套基于JAVA的炒粉信息管理系统后端,该系统旨在\textbf{提升炒粉店的管理效率、优化客户体验以及实现数据的精准分析}。项目涉及\textbf{订单管理、库存管理、员工管理、数据分析}等多个核心功能模块的设计与实现。
        \item[\Large $\bullet$] \small 涉及技术:SpringBoot、Mybatis、SpringMVC、Maven、Git、Linux、Redis等
        \item[\Large $\bullet$] \small 职责描述: \textbf{这里遵循从短到长的原则}
        \begin{enumerate}
            \item 需求分析: 深入分析了炒粉店的管理需求,梳理出后端系统应支持的核心功能,并与团队成员共同制定了详细的需求文档。
            \item 系统设计: 负责系统后端架构的设计,包括\textbf{数据库设计、接口设计、模块划分}等。通过合理的架构设计,确保了系统的稳定性与可扩展性。
            \item 编码实现: 使用Java语言及Spring Boot框架进行后端开发,实现了\textbf{订单管理、库存管理、员工管理}等核心功能模块的接口。同时,\textbf{优化了数据库操作,提升了数据访问效率}。
        \end{enumerate}
        \item[\Large $\bullet$] \small 项目成就:获得\textbf{坤算机软件著作权},获得公司实习转正机会、掌握\textbf{Java炒粉核心技术}等等
        
    \end{itemize}

    \vspace{1em}

    \textbf{\uppercase\expandafter{\romannumeral3}. 坤成公司: 两年半练习生
    \quad{\color{CQMU_Green}\faMapMarker\textit{哎哟你在哪}}}  \hfill 2024年6月-2025年2月
    \vspace{0.4em}
    \begin{itemize}[itemsep=1pt, leftmargin=2em]
        \item[\Large $\bullet$] \small 你已小成;
        \item[\Large $\bullet$] \small 但学无止境,继续向前吧。
    \end{itemize}

    \vspace{1em}

    \newpage

 % ------------------------------------------------------------------------
	\begin{tikzpicture}[remember picture, overlay]
		\node[anchor=north, inner sep=0pt](header) at (current page.north){
			\includegraphics[width=1\paperwidth]{images/cqmu-header-green.png}
		};
		\node[anchor=west](school_logo) at (header.west){
			\hspace{0.5cm}
			\includegraphics[width=0.35\textwidth]{images/cqmu-logo-white.png}
		};
		\node[anchor=east](school_name) at(header.east){
			\textcolor{white}{\textbf{\school}}
			\hspace{0.5cm}
		};
	\end{tikzpicture}
	\vspace{-4em}

	% 页脚,联系方式
	\begin{tikzpicture}[remember picture, overlay]
		\node[anchor=south, inner sep=0pt](footer) at (current page.south){
			\includegraphics[width=\paperwidth]{images/cqmu-footer.png}
		};
        % 联系方式
        \node[anchor=center] at(footer.center){\contact};
	\end{tikzpicture}
	
	% 背景
	\begin{tikzpicture}[remember picture, overlay]
		\node[opacity=0.1] at(current page.center){
			\includegraphics[width=0.8\paperwidth, keepaspectratio]{images/cqmu-logo.eps}
		};
	\end{tikzpicture}
% ---------------------------------------------------------------------
    
    
    \subsection{\Large 项目经验}
    \textbf{\uppercase\expandafter{\romannumeral1}.} \textbf{学生管理系统}\hspace{1em}项目负责人/成员/打工人
     \hfill 2024年1月-2023年12月

    
    % \vspace{0.4em}
    \begin{itemize}[itemsep=1pt, leftmargin=2em]
        \item[\Large $\bullet$] \small 项目简介: 学生管理系统是一款\textbf{旨在提高学校学生信息管理效率}的软件。该系统具备\textbf{学生信息管理、成绩管理、课程管理、考勤管理}等功能,通过现代化的信息管理手段,帮助学校实现对学生信息的全面、准确、快速管理。
        \item[\Large $\bullet$] \small 责任描述: 
        \begin{enumerate}
            \item 需求分析: 负责收集学校各部门对于学生管理系统的功能需求,\textbf{整理并撰写需求规格说明书}。与团队成员、学校相关部门沟通,确保需求明确、无歧义。
            \item 系统设计: 根据需求分析结果,\textbf{设计系统的整体架构、数据库结构、界面布局}等。\textbf{绘制系统流程图、数据流程图、数据库设计图}等相关设计文档。
            \item 编码实现:  负责编写系统后台代码,\textbf{包括数据库操作、业务逻辑处理}等。与前端开发人员协作,确保前后端数据交互顺畅。同时,对代码进行测试,确保系统的\textbf{稳定性和安全性}。
        \end{enumerate}
        \item[\Large $\bullet$] \small 项目成就: 成功开发并上线学生管理系统,网址:\hyperlink{cqmu}{https://www.cqmu.edu.cn/},得到学校师生的广泛好评。系统运行稳定,提高了学校学生信息管理的效率。通过此项目,团队成员积累了宝贵的开发经验,提升了团队协作能力。该系统被评为学校优秀项目,并在相关比赛中获得奖项......

    \end{itemize}

    \vspace{1em}
    \textbf{\uppercase\expandafter{\romannumeral2}.} \textbf{大模型医学影像学应用} \hfill \textbf{临床二期实验} \\
    项目负责人/打工人/成员 \hfill 2023年12月-2024年1月\\
    \textbf{独自进行前期研究,实验验证和论文写作}。重点探索了将\textbf{多模态医学影像}(如MRI和CT)与\textbf{自然语言模型}相结合的方法,以提高脑肿瘤检测的准确性。成功将对比学习和Prompt Fine-Tuning技术应用于医学影像分析中,将模型在脑肿瘤检测任务中的性能提升到了\textbf{超越SOTA的水平},特别是在 one-shot 场景下取得了显著的改进。

    % 不知道写啥好
    \vspace{1em}                % 这是换行用的


    % \subsection{基于阴间蓝牙通信系统的高速运转机器设计\hfill SCI期刊-二区在投}
    \textbf{\uppercase\expandafter{\romannumeral3}.} \textbf{基于阴间蓝牙通信系统的高速运转机器设计\hfill SCI期刊-二区在投}\\
    \textbf{第一作者} \hfill 2023年12月-2024年1月
    
    此处省略一万字。


    \subsection{\Large 社团活动}
    \textbf{\uppercase\expandafter{\romannumeral1}.} \textbf{学习部}\hspace{1em} 负责人 \hfill 2023年1月-2024年1月\\

    学习部作为一个专注于学术、学习支持和文化活动组织的部门,在促进学生全面发展、提高学习效率和营造良好学习氛围方面发挥着重要作用
    
    \vspace{1em}

    
    \textbf{\uppercase\expandafter{\romannumeral2}.} \textbf{计算机社团}\hspace{1em}你的职务 \hfill 2023年1月-2024年1月\\

    计算机社团为学生提供了一个学习计算机知识和技能的平台,通过组织各类技术讲座、研讨会、培训课程等活动,帮助学生深入了解计算机领域的最新技术和发展趋势。
    
    \vspace{1em}

    
    \textbf{\uppercase\expandafter{\romannumeral3}.} \textbf{一些活动}\hspace{1em}你的身份 \hfill 2023年1月-2024年1月\\ 

    大学活动通常包括学术讲座、研讨会、研讨会等,这些活动能够让学生接触到最新的学术研究成果,拓宽学术视野,加深对专业知识的理解。


    
    % 这是个工科类加粗的例子
    % \vspace{1em}                % 这是换行用的
    % \subsection{Improving Multi-Modal Brain Tumor Detection with Contrastive Learning and CLIP Prompt Fine-Tuning \hfill EI会议-在投}

    
    % \subsection{基于阴间蓝牙通信系统的高速运转机器设计\hfill SCI期刊-二区在投}
        
    % \textbf{一作} \hfill 2023年12月-2024年1月
    
    % 你有这么\textbf{高速运转的机械}进入中国,进入我给出的原理,小时候。就是\textbf{研发}人,就是研发这个东西的原理是\textbf{阴间政权}管。你知道为什么有圣灵给它\textbf{运转仙位}?还有专门饲养这个?为什么地下产这种东西?他管的他是五世同堂旗下子孙。你以为我跟你闹着玩儿呢?你不警察吗?黄龙江一带全都\textbf{带蓝牙}。黄龙江我告诉你在阴间是是那个化名,化名我小舅,亲小舅,张学兰的那个嫡子、嫡孙。

    % % 在研的也能写
    % % 课程大作业也能写,但是不要标明是大作业就行
    % % 一个例子,别真写金工实习做榔头
    % \vspace{1em}
    % \subsection{大国工匠-论锤子是怎么炼成的\hfill 工程项目-已完结}
        
    % \textbf{主要技术负责人} \hfill 忘了日期
    
    % 对古代中国工匠制作锤子的历史、工艺和技术进行深入研究,包括原材料的选择、工具的制作、锤子的设计和锻造工艺等方面。结合现代工艺技术和设计理念,探索如何将古代工匠的技艺与现代制造工艺相结合,以提高锤子的性能、品质和设计。通过实际制作锤子的过程,验证理论和技术的可行性,并对制作过程中的关键环节进行深入分析和总结。

    % 终于用完一页了,加一页展示怎么加页眉页脚
    % \newpage

    % 宝子,请加在这里
	% 页眉,校徽,学院名


    % 技能特长,上面写很多的话,这里就随便写点,反正上面都看出来了。上面写的不多的话,这里着重强调你会什么。
    % 哦,你找工作的话,这里多写点,记得对口,可以\textbf{}加粗。
    % 这里能吹牛皮就吹牛皮,但是确保面试的时候别露馅就行。
\newpage
    
 % ------------------------------------------------------------------------
	\begin{tikzpicture}[remember picture, overlay]
		\node[anchor=north, inner sep=0pt](header) at (current page.north){
			\includegraphics[width=1\paperwidth]{images/cqmu-header-green.png}
		};
		\node[anchor=west](school_logo) at (header.west){
			\hspace{0.5cm}
			\includegraphics[width=0.35\textwidth]{images/cqmu-logo-white.png}
		};
		\node[anchor=east](school_name) at(header.east){
			\textcolor{white}{\textbf{\school}}
			\hspace{0.5cm}
		};
	\end{tikzpicture}
	\vspace{-4em}

	% 页脚,联系方式
	\begin{tikzpicture}[remember picture, overlay]
		\node[anchor=south, inner sep=0pt](footer) at (current page.south){
			\includegraphics[width=\paperwidth]{images/cqmu-footer.png}
		};
        % 联系方式
        \node[anchor=center] at(footer.center){\contact};
	\end{tikzpicture}
	
	% 背景
	\begin{tikzpicture}[remember picture, overlay]
		\node[opacity=0.1] at(current page.center){
			\includegraphics[width=0.8\paperwidth, keepaspectratio]{images/cqmu-logo.eps}
		};
	\end{tikzpicture}
% ---------------------------------------------------------------------
    
    \section{\makebox[\widthof{\faWrench}][c]{\color{CQMU_Green}{\faWrench}}\quad 专业技能}
    \vspace{0.5em}
    \begin{itemize}
        \item 这里就先强调你最擅长什么,\textbf{适当}吹牛可以,能稳得住就行,下面是一个硬件狗的自述:
        \item 熟练使用\textbf{\Cpp 、Python、Matlab}编程语言。
        \item 熟悉Windows与Linux端开发。
        \item 熟练使用\textbf{Tensorflow,Pytorch}等深度学习框架。
        \item 熟练掌握\Cpp 与Python环境下OpenCV与Qt应用的开发,且\textbf{熟练使用Qt Creator}软件。
        \item \textbf{熟练使用Altium Designer与LCEDA}进行封装绘制与板子设计。
        \item 熟练使用Keil,Arduino IDE等集成开发软件。
        \item 了解模式识别,强化学习,遗传算法,知识蒸馏等相关概念。
    \end{itemize}
    
    \section{\makebox[\widthof{\faTrophy}][c]{\color{CQMU_Green}{\faTrophy}}\quad 竞赛经历}
    \vspace{-1em}
    % 这里就主要写一些竞赛获奖,可以按照含金量,获奖等级这些维度来排序(如果没有市级及以上的话,就写点学校的吧,赶快呼吁学弟学妹们多多参与下专业比赛,不然到时写简历可就两行泪了)
    \begin{table}[h!]
        \begin{tabularx}{\textwidth}{Xp{\widthof{第零负责人}}p{\widthof{国家级-第100名}}p{\widthof{2030年13月}}}
            \textbf{2023年全国大学生无聊透顶比赛} & 第一负责人 & 国家级-第10名 & 2023年4月 \\
            \textbf{2023年优质睡眠大赛} & 个人参赛 & 国家级-一等奖 & 2023年8月\\
            \textbf{2022年全国大学生干饭大赛} & 个人参赛 & 省级-一等奖 & 2022年12月\\
            % 同理,可以自己加
        \end{tabularx}
    \end{table}

    \section{\makebox[\widthof{\faCertificate}][c]{\color{CQMU_Green}{\faCertificate}}\quad 技能证书}
    \vspace{0.5em}
    \begin{itemize}
        \item 与你求职的岗位招聘要求相关的证书,越相关越OK,比如你要面网络工程师:
        \item 2021.6 \hspace{2em} Cisco Certified Architect Certification
        \item 2021.6 \hspace{2em} Huawei Certified ICT Expert-Security
        \item 2021.6 \hspace{2em} 软考-软件设计师中级
        \item 2021.6 \hspace{2em} 计算机四级证书
    \end{itemize}

    % 所获荣誉(这个看你想不想写了)
    \section{\makebox[\widthof{\faStar}][c]{\color{CQMU_Green}{\faStar}}\quad 荣誉与奖励}
    \vspace{-1em}
    \begin{multicols}{2}
        \begin{itemize}
            \item 2021.6 \hspace{2em} 某先进个人 
            \item 2021.6 \hspace{2em} 某罕见奖学金获得者 
            \item 2021.6 \hspace{2em} 某优秀志愿者
            \item 2021.6 \hspace{2em} 某二课堂之星
            \item 2021.6 \hspace{2em} 某大使
            \item 2021.6 \hspace{2em} 某年某奖学金某等奖
            \item 2021.6 \hspace{2em} 某年优秀团员称号
            \item 2021.6 \hspace{2em} 某年某称号
        \end{itemize}
    \end{multicols}

    % 自我评价
    \section{\makebox[\widthof{\faStreetView}][c]{\color{CQMU_Green}{\faStreetView}}\quad 自我评价}
    \vspace{0.5em}
    \begin{itemize}
        \item 个人特质:诚实、责任感、团队合作等
        \item 职业规划:简短说明你的职业目标和未来计划
    \end{itemize}

\newpage
    
 % ------------------------------------------------------------------------
	\begin{tikzpicture}[remember picture, overlay]
		\node[anchor=north, inner sep=0pt](header) at (current page.north){
			\includegraphics[width=1\paperwidth]{images/cqmu-header-green.png}
		};
		\node[anchor=west](school_logo) at (header.west){
			\hspace{0.5cm}
			\includegraphics[width=0.35\textwidth]{images/cqmu-logo-white.png}
		};
		\node[anchor=east](school_name) at(header.east){
			\textcolor{white}{\textbf{\school}}
			\hspace{0.5cm}
		};
	\end{tikzpicture}
	\vspace{-4em}

	% 页脚,联系方式
	\begin{tikzpicture}[remember picture, overlay]
		\node[anchor=south, inner sep=0pt](footer) at (current page.south){
			\includegraphics[width=\paperwidth]{images/cqmu-footer.png}
		};
        % 联系方式
        \node[anchor=center] at(footer.center){\contact};
	\end{tikzpicture}
	
	% 背景
	\begin{tikzpicture}[remember picture, overlay]
		\node[opacity=0.1] at(current page.center){
			\includegraphics[width=0.8\paperwidth, keepaspectratio]{images/cqmu-logo.eps}
		};
	\end{tikzpicture}
% ---------------------------------------------------------------------

    % 其他(也是看你想不想写)
    \section{\makebox[\widthof{\faInfo}][c]{\color{CQMU_Green}{\faInfo}}\quad 其他}
    \vspace{0.5em}
    \begin{itemize}
        \item 英语水平-CET6级x分,雅思y分,托福z分
        \item 头发茂密程度
        \item 技术博客: 某网址
        \item 一些关系不大的证书,可写可不写
        \item 教师资格证:xxx
        \item 普通话证书:几级几等
        \item 文字排版:\LaTeX
        \item 兴趣爱好:简短列出,展示个性
        \item 推荐人:如果有,可以列出推荐人的联系信息,通常需要事先征得推荐人同意
    \end{itemize}

\end{document}
